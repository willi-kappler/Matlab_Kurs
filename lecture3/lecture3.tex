\documentclass{beamer}

\usepackage[utf8]{inputenc}
\usepackage[T1]{fontenc}
\usepackage{textpos}
\usepackage{tikz}

\usetheme{Madrid}
\usecolortheme{beaver}

% Custom changes:
\setbeamertemplate{footline}[frame number]{}
\definecolor{university_tuebingen}{RGB}{165,30,55}
\setbeamercolor{frametitle}{fg=university_tuebingen, bg=white}
\setbeamercolor{title}{fg=university_tuebingen}

\addtobeamertemplate{frametitle}{}{
\begin{tikzpicture}[remember picture, overlay]
\node[anchor=north east,yshift=0cm] at (current page.north east)
{\includegraphics[width=3cm]{../common/logo_uni_tuebingen2.png}};
\end{tikzpicture}}



\author{Prof. Dr. Christiane Zarfl, Dipl.-Inf. Willi Kappler}
%\date{\today}
\date{}
%\institute{Universität Tübingen}
\institute{}


\matlabTitle{3. Matrizen}

% Rechtschreibprüfung mit: aspell -l de -t --tex-check-comments -c lecture3/lecture3.tex

\setcounter{mchapter}{3}
\setcounter{mexercise}{0}

\begin{document}
  {
\beamertemplatenavigationsymbolsempty % suppress navigation on this (= first) slide
\begin{frame}[plain] % plain means: no header and footer on this (= first) slide
    \begin{textblock*}{0cm}(0.5cm, -0.7cm)
        \includegraphics[width=11.0cm]{common/logo_uni_tuebingen.png}
    \end{textblock*}
    \titlepage
    \begin{textblock*}{0cm}(0.1cm, -2.5cm)
        \textcolor{university_tuebingen}{\rule{11.8cm}{0.2cm}}
    \end{textblock*}
\end{frame}
}


    \section{Einleitung}

    \subsection{Motivation}
    \begin{frame}
        \frametitle{Sie wissen bereits...}
        \begin{itemize}
            \item wie Sie in Matlab mit Vektoren rechnen.
            \item wie Sie Ergebnisse/Daten in 2D-Grafiken darstellen.
            \item wie Sie durch Scripte/Programme Befehlsfolgen wiederverwertbar machen.
        \end{itemize}

        \textit{Wie kann ich mit Matrizen rechnen?} \\
        \textit{Und wie kann ich 3D-Grafiken erstellen?}
    \end{frame}

    \begin{frame}
        \frametitle{Nach diesem dritten Block...}
        \begin{itemize}
            \item rechnen Sie sicher mit Matrizen.
            \item können Sie lineare Gleichungssysteme durch in Matlab implementierte Matrizenfunktionen einfach berechnen.
            \item können Sie zweidimensionale Daten darstellen:
            \begin{itemize}
                \item in einer 3D-Abbildung
                \item in einem Oberflächenplot (``Höhenlinien'')
            \end{itemize}
        \end{itemize}
    \end{frame}

    \section{Matrizen}

    \subsection{Matrizen}
    \begin{frame}
        \frametitle{Matrizen}
        \begin{itemize}
            \item Matrizen sind zweidimensionale Zahlenfelder, die man wie Tabellen schreiben kann.
            \item Matrizen sind nützlich für:
            \begin{itemize}
                \item die Beschreibung tensorieller Eigenschaften  (z.B. optischer Brechungsindex, magnetische Suszeptibilität, Spannungstensor).
                \item die Formulierung linearer Gleichungssysteme.
                \item die Diskretisierung zweidimensionaler Daten
            \end{itemize}
        \end{itemize}
    \end{frame}

    \subsection{LGS}
    \begin{frame}
        \frametitle{Beispiel: lineares Gleichungssystem}

        \begin{align*}
            3x_{1} + 4x_{2} - 2x_{3}  &= 4 \\
            -1x_{1} + 2x_{2} + 8x_{3} &= -1 \\
            2x_{1} - 5x_{3}  &= 3
        \end{align*}

        \begin{itemize}
            \item In Matrix-Vektor Schreibweise: $Ax = b$, Formales Lösen: $x = A^{-1}b$
            \item Definition von Matrix $A$ in Matlab:
            \item \matlabInput{A = [3,4,-2; -1,2,8; 2,0,-5];}, Semikolon in eckiger Klammer: neue Zeile
            \item Definition von Spaltenvektor b in Matlab:
            \item \matlabInput{b = [4; -1; 3];}, Semikolon in eckiger Klammer: neue Zeile
        \end{itemize}
    \end{frame}

    \begin{frame}
        \frametitle{LGS Fortsetzung}
        \begin{itemize}
            \item Invertieren der Matrix $A$ mit \matlabInput{\matlabLink{inv}(A)}: \matlabInput{x = \matlabLink{inv}(A)*b}
            \item Das Invertieren einer Matrix ist ``nummerisch teurer'' als das Lösen eines einzelnen Gleichungssystems.
            \item Hierfür hat Matlab den Backslash: \matlabInput{x = A \string\ b} bedeutet ``löse $Ax = b$ nach $x$''
        \end{itemize}
    \end{frame}

    \subsection{Transponieren}
    \begin{frame}
        \frametitle{Transponieren, Zeilen- und Spaltenvektoren}
        \begin{itemize}
            \item Transponieren = Vertauschen von Zeilen und Spalten
            \item Wird in Matlab mit einem Apostroph hinter der Variable durchgeführt
            \item Bsp.: Definiere b als Zeilenvektor und wandle es in Spaltenvektor um:
            \item \matlabInput{b = [4, -1, 3]; b = b'}, (Ein Gleichheitszeichen ist eine Zuweisung)
            \item \alert{Achtung!} Das Gleichungssystem Ax = b kann nur gelöst werden, wenn
            \begin{itemize}
                \item $A$ eine quadratische, invertierbare Matrix ist (die Determinante ist nicht null oder $A$ hat vollen Rang)
                \item $b$, ein Spaltenvektor, hat die selbe Zeilenanzahl wie Matrix $A$
                \item 90\% der Fehler in Anwendungen der linearen Algebra (Vektor- und Matrixrechnungen) sind falsche Dimensionierungen,
                meistens wegen fehlender Transponierung
            \end{itemize}
        \end{itemize}
    \end{frame}

    \subsection{Elementweiser Zugriff}
    \begin{frame}
        \frametitle{Zugriff auf Elemente einer Matrix}
        \begin{itemize}
            \item Der Zugriff auf einzelne Elemente wird durch runde Klammern angegeben: \matlabInput{A(3,2)} gibt mir den Inhalt des Elements
            in der \textbf{dritten Zeile} und \textbf{zweiten Spalte} von Matrix $A$.
            \item Ein Doppelpunkt als Index heißt ``alle'' (s. \matlabInput{doc \matlabLink{colon}}):
            \begin{itemize}
                \item \matlabInput{A(2,:)} gibt alle Werte der \textbf{2. Zeile} aus (das ist ein Zeilenvektor)
                \item \matlabInput{A(:,1)} gibt alle Werte der \textbf{1. Spalte} aus (das ist ein Spaltenvektor)
            \end{itemize}
            \item Man kann auch Werte eines bestimmten Elementes einer Matrix zuweisen: \matlabInput{A(3,2) = 7} weist dem Element in der dritten
            Zeile und zweiten Spalte den Wert 7 zu.
        \end{itemize}
    \end{frame}

    \begin{frame}
        \frametitle{Fortsetzung}
        \begin{itemize}
            \item Eine absolute leere Matrix (keine Zeilen, keine Spalten) kann durch leere eckige Klammern erzeugt werden:
            \begin{itemize}
                \item \matlabInput{B = [];} erzeugt eine leere Matrix
                \item \matlabInput{A(:,1) = [ ]} bewirkt, dass die erste Spalte von Matrix A gelöscht wird
            \end{itemize}
            \item \matlabInput{\matlabLink{size}(A)} gibt die Dimension der Matrix wieder. Das Ergebnis ist ein Vektor \matlabOutput{[Zeilenzahl, Spaltenzahl]}
        \end{itemize}
    \end{frame}

    \subsection{Spezielle Matrizen}
    \begin{frame}
        \frametitle{Erzeugung spezieller Matrizen}
        \begin{itemize}
            \item \matlabInput{\matlabLink{zeros}(2)}: ergibt eine 2$\times$2 Nullmatrix
            \item \matlabInput{\matlabLink{zeros}(2,4)}: ergibt eine 2$\times$4 Nullmatrix
            \item \matlabInput{\matlabLink{eye}(3)}: ergibt eine 3$\times$3 Einheitsmatrix (Einsen auf der Hauptdiagonalen)
            \item \matlabInput{\matlabLink{ones}(3)}: ergibt eine 3$\times$3 Matrix aus lauter Einsen
            \item \matlabInput{\matlabLink{rand}(2,3)}: ergibt eine 2x3 Matrix mit Zufallszahlen (gleichverteilt zwischen 0 und 1)
            \item ...
        \end{itemize}
    \end{frame}

    \secMexercise
    \begin{frame}
        \frameMexercise
        \begin{exercise}
            \sloppy
            Geben Sie folgende Matrix ein:\\
            \begin{displaymath}
                A =
                \begin{pmatrix}
                    1 & 3 & 5 \\
                    2 & 0 & 1 \\
                    2 & 4 & 6
                \end{pmatrix}
            \end{displaymath}
             Was ist das Ergebnis der folgenden Anweisungen? (Überlegen Sie, bevor Sie mit Matlab rechnen.) \\

             \begin{columns}[t]
               \begin{column}{5cm}
                 \texttt{B = A'} \keys{\return} \\
                 \texttt{A(1,:)} \keys{\return} \\
                 \texttt{A(:,3)} \keys{\return} \\
                 \texttt{A(2,2)} \keys{\return} \\
               \end{column}
               \begin{column}{5cm}
                   \texttt{B(2,2)} \keys{\return} \\
                   \texttt{B(3,2)} \keys{\return} \\
                   \texttt{size(A), size(B)} \keys{\return}
               \end{column}
             \end{columns}
        \end{exercise}
    \end{frame}

    \secMexercise
    \begin{frame}
        \frameMexercise
        \begin{exercise}
            \sloppy
            Ersetzen Sie in einer 3$\times$4 Matrix $B$ mit Zufallszahlen das erste Element in der ersten Zeile durch 0 und
            streichen Sie die gesamte zweite Spalte von $B$.
        \end{exercise}
    \end{frame}

    \secMexercise
    \begin{frame}
        \frameMexercise
        \begin{exercise}
            \sloppy
            Erzeugen Sie eine 2$\times$5 Matrix $A$ mit Zufallszahlen. Bilden Sie daraus:\\
            \begin{enumerate}
                \item eine Matrix $B$, die aus den ersten drei Spalten von $A$ besteht
                \item eine Matrix $C$, die aus der zweiten und vierten Spalte von $A$ besteht
            \end{enumerate}

            \vspace{0.5cm}

            Finden Sie die dazu nötigen Anweisungen mit \texttt{help colon} und \texttt{help paren} selbst heraus.
        \end{exercise}
    \end{frame}

    \subsection{Matrixoperationen}
    \begin{frame}
        \frametitle{Matrixoperationen}
        \begin{itemize}
            \item \matlabInput{A+B, A-B, A*B} bezeichnen die Matrizenaddition, -subtraktion und -multiplikation.
            \item \alert{Achtung!} \matlabInput{A*B} ist wirklich eine Matrixmultiplikation, die elementweise Multiplikation ist \matlabInput{A.*B}
            \item Vergleiche:
            \begin{itemize}
                \item \matlabInput{[1 2; -2 5] .* [3 6; 0 -1]}
                \item \matlabInput{[1 2; -2 5] * [3 6; 0 -1]}
            \end{itemize}
            (Fortsetzung nächste Folie)
        \end{itemize}
    \end{frame}

    \begin{frame}
        \frametitle{Fortsetzung}
        \begin{itemize}
            \item Für \textbf{quadratische} Matrizen:
            \item \matlabInput{\matlabLink{det}(A)} ist die Determinante der Matrix $A$
            \item \matlabInput{\matlabLink{eig}(A)} berechnet die Eigenwerte
            \item \matlabInput{\matlabLink{inv}(A)} ist die Inverse der Matrix
            \item Ein kleiner Test der Operationen:
            \begin{itemize}
                \item \matlabInput{A = \matlabLink{rand}(3)}
                \item \matlabInput{\matlabLink{det}(A)}
                \item \matlabInput{e = \matlabLink{eig}(A), \matlabLink{prod}(e)}
                \item \matlabInput{iA = \matlabLink{inv}(A)}
                \item \matlabInput{1/\matlabLink{det}(iA)}
            \end{itemize}
        \end{itemize}
    \end{frame}

    \secMexercise
    \begin{frame}
        \frameMexercise
        \begin{exercise}
            \sloppy
            Sei \texttt{A = eye(3,3), B = [1 2 3; 4 5 6; 7 8 9]}. Was ist das Ergebnis von: \\
            \begin{columns}[t]
              \begin{column}{5cm}
                \texttt{A+B} \keys{\return} \\
                \texttt{A*2} \keys{\return} \\
                \texttt{A/2} \keys{\return} \\
              \end{column}
              \begin{column}{5cm}
                  \texttt{B.\string^2} \keys{\return} \\
                  \texttt{A.*B} \keys{\return} \\
                  \texttt{A*B} \keys{\return}
              \end{column}
            \end{columns}

            \vspace{0.5cm}

            Überlegen Sie, bevor Sie mit Matlab rechnen!
        \end{exercise}
    \end{frame}

    \secMexercise
    \begin{frame}
        \frameMexercise
        \begin{exercise}
            \sloppy
            Lösen Sie das Gleichungssystem:

            \begin{align*}
                x + 4y + 6z  &= 5 \\
                3x - z &= 7 \\
                2x + 7y -3z &= 1
            \end{align*}

        \end{exercise}
    \end{frame}

    \section{Visualisierung}

    \subsection{Zweidimensionale Funktionen}
    \begin{frame}
        \frametitle{Visualisierung zweidimensionaler Funktionen}
        \begin{itemize}
            \item Aufgabe: Visualisierung einer Funktion \texttt{z = f(x,y)}, die von zwei Koordinaten abhängt
            \item Beispiel: \texttt{z = cos(x*y)}
            \item Herangehensweise:
            \begin{enumerate}
                \item Erzeuge ein regelmäßiges Gitter von x- und y-Werten (Vektoren \texttt{X} und \texttt{Y})
                \item Führe elementweise Berechnung von \texttt{z} durch (erzeuge eine Matrix \texttt{Z} mit den z-Werten zu jedem x,y-Paar)
                \item Plotte die Daten mit speziellen Befehlen
            \end{enumerate}
        \end{itemize}
    \end{frame}

    \begin{frame}
        \frametitle{Fortsetzung}
        In Matlab:
        \begin{enumerate}
            \item Regelmäßiges Gitter:
            \begin{itemize}
                \item Erzeuge Vektoren der regelmäßigen x- und y-Werte: \matlabInput{xvec = [-3:0.1:3]; yvec = [-3:0.1:3];}
                \item Erzeuge hieraus ein regelmäßiges Gitter mit \matlabInput{\matlabLink{meshgrid}}: \matlabInput{[X,Y] = \matlabLink{meshgrid}(xvec,yvec);}
            \end{itemize}
            \item Jetzt \textit{elementweise} Berechnung: \matlabInput{Z = cos(X.*Y);}
        \end{enumerate}
    \end{frame}

    \subsection{Möglichkeiten in 2D und 3D}
    \begin{frame}
        \frametitle{Visualisierungsmöglichkeiten}
        \begin{itemize}
            \item 3D-Visualisierung als Gitternetz: \matlabInput{\matlabLink{mesh}}
            \insertMatlabOnlyCode{example3_1}
            \item 3D-Visualisierung als Oberfläche: \matlabInput{\matlabLink{surf}}
            \insertMatlabOnlyCode{example3_2}
            \item Testen Sie, was \matlabInput{\matlabLink{shading} interp} bewirkt
        \end{itemize}
    \end{frame}

    \begin{frame}
        \frametitle{Fortsetzung}
        2D-Visualisierung mit Isolinien und -flächen: \matlabInput{\matlabLink{contour}} und \matlabInput{\matlabLink{pcolor}}
        \begin{itemize}
            \item Erzeugen einer Karte mit 50 Höhenlinien: z.B.: \matlabInput{contour(X,Y,Z,50);}
            \item Erzeugen einer Karte mit gefüllten Flächen, die Höhenwerte darstellen: \matlabInput{pcolor(X,Y,Z); shading interp}
            \item Eine Farblegende gibt es mit \matlabInput{\matlabLink{colorbar}}
        \end{itemize}
    \end{frame}

    \secMexercise
    \begin{frame}
        \frameMexercise
        \begin{exercise}
            \sloppy
            Stellen Sie die lineare Funktion $z = -0.4x -0.8y + 3$ für $0 \leq x \leq 5$ und $0 \leq y \leq 5$ graphisch dar. \\ \\

            Um welche Art von Fläche handelt es sich?
        \end{exercise}
    \end{frame}

    \secMexercise
    \begin{frame}
        \frameMexercise
        \begin{exercise}
            \sloppy
            Stellen Sie die Funktion $z = x^{2} + y^{2}$ für $-3 \leq x \leq 3$ und $-3 \leq y \leq 3$ graphisch dar. \\ \\
            Verwenden Sie \texttt{subplot} um mehrere Darstellungsmöglichkeiten in einer Grafik zu vergleichen
        \end{exercise}
    \end{frame}

    \section{Ausblick}
    \subsection{Ausblick}
    \begin{frame}
        \frametitle{Ausblick}
        \begin{itemize}
            \item Wie kann ich häufig vorkommende Berechnungen/Abläufe automatisieren?
        \end{itemize}
    \end{frame}


\end{document}
