\documentclass{beamer}

\usepackage[utf8]{inputenc}
\usepackage[T1]{fontenc}
\usepackage{textpos}
\usepackage{tikz}

\usetheme{Madrid}
\usecolortheme{beaver}

% Custom changes:
\setbeamertemplate{footline}[frame number]{}
\definecolor{university_tuebingen}{RGB}{165,30,55}
\setbeamercolor{frametitle}{fg=university_tuebingen, bg=white}
\setbeamercolor{title}{fg=university_tuebingen}

\addtobeamertemplate{frametitle}{}{
\begin{tikzpicture}[remember picture, overlay]
\node[anchor=north east,yshift=0cm] at (current page.north east)
{\includegraphics[width=3cm]{../common/logo_uni_tuebingen2.png}};
\end{tikzpicture}}



\author{Prof. Dr. Christiane Zarfl, Dipl.-Inf. Willi Kappler}
%\date{\today}
\date{}
%\institute{Universität Tübingen}
\institute{}


\title{{
\beamertemplatenavigationsymbolsempty % suppress navigation on this (= first) slide
\begin{frame}[plain] % plain means: no header and footer on this (= first) slide
    \begin{textblock*}{0cm}(0.5cm, -0.7cm)
        \includegraphics[width=11.0cm]{common/logo_uni_tuebingen.png}
    \end{textblock*}
    \titlepage
    \begin{textblock*}{0cm}(0.1cm, -2.5cm)
        \textcolor{university_tuebingen}{\rule{11.8cm}{0.2cm}}
    \end{textblock*}
\end{frame}
}
\\{\scriptsize 2. Der Editor und Grafikplots}}

% Rechtschreibprüfung mit: aspell -l de -t --tex-check-comments -c lecture2/lecture2.tex

\setcounter{mexercise}{0}

\begin{document}
  {
\beamertemplatenavigationsymbolsempty % suppress navigation on this (= first) slide
\begin{frame}[plain] % plain means: no header and footer on this (= first) slide
    \begin{textblock*}{0cm}(0.5cm, -0.7cm)
        \includegraphics[width=11.0cm]{common/logo_uni_tuebingen.png}
    \end{textblock*}
    \titlepage
    \begin{textblock*}{0cm}(0.1cm, -2.5cm)
        \textcolor{university_tuebingen}{\rule{11.8cm}{0.2cm}}
    \end{textblock*}
\end{frame}
}


    \section{Einleitung}

    \subsection{Motivation}
    \begin{frame}
        \frametitle{Sie wissen bereits...}
        \begin{itemize}
            \item wie Sie in Matlab Variablen definieren und damit Werte wiederverwenden können.
            \item wie Sie in Matlab mit Vektoren rechnen und damit effizient auch große Datenmengen verarbeiten können.
        \end{itemize}

        \vspace{0.3cm}

        Wie kann man mehrere Rechenschritte/Matlab-Befehle, die häufig benötigt werden, ``speichern'' und zusammenfassen? \\

        \vspace{0.3cm}

        Nach diesem zweiten Block... \\

        \vspace{0.3cm}

        \begin{itemize}
            \item können Sie eigene Script Files erstellen (Bsp. 1D-Stofftransport- gleichung)
            \item können Sie x-y Plots erstellen, bearbeiten und speichern.
        \end{itemize}
    \end{frame}


    \subsection{Script Files}
    \begin{frame}
        \frametitle{Programme Schreiben mit Matlab - Script Files}
        \begin{itemize}
            \item Matlab ist mehr als nur ein Taschenrechner
            \item Ein Script-File ist eine Aneinanderreihung von Matlab-Befehlen
            \item Matlab-Scripts haben die Dateiendung ``\texttt{.m}''
            \item Bsp.: Ihr Programm heißt \texttt{MeinScript.m} und befindet sich im Verzeichnis \texttt{MyDocuments\textbackslash ich\textbackslash Matlab}
            \item Dieser Pfad muss das gegenwärtige \textbf{Arbeitsverzeichnis} von Matlab sein
            \item Dann führt der Befehl \matlabInput{MeinScript} (ohne Dateiendung) im ``Command Window'' den Inhalt des Scriptes aus
            \item Dateinamen dürfen keine Sonderzeichen oder Leerzeichen enthalten und auch nicht mit Ziffern anfangen
            \item Script Files/Programme können mit jedem Texteditor oder mit dem Matlab-eigenen Editor erstellt werden
          \end{itemize}
      \end{frame}

\end{document}
