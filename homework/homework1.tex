\documentclass{beamer}

\usepackage[utf8]{inputenc}
\usepackage[T1]{fontenc}
\usepackage{textpos}
\usepackage{tikz}

\usetheme{Madrid}
\usecolortheme{beaver}

% Custom changes:
\setbeamertemplate{footline}[frame number]{}
\definecolor{university_tuebingen}{RGB}{165,30,55}
\setbeamercolor{frametitle}{fg=university_tuebingen, bg=white}
\setbeamercolor{title}{fg=university_tuebingen}

\addtobeamertemplate{frametitle}{}{
\begin{tikzpicture}[remember picture, overlay]
\node[anchor=north east,yshift=0cm] at (current page.north east)
{\includegraphics[width=3cm]{../common/logo_uni_tuebingen2.png}};
\end{tikzpicture}}



\author{Prof. Dr. Christiane Zarfl, Dipl.-Inf. Willi Kappler}
%\date{\today}
\date{}
%\institute{Universität Tübingen}
\institute{}


\matlabTitle{Hausaufgaben 1}

% Rechtschreibprüfung mit: aspell -l de -t --tex-check-comments -c lecture6/lecture6.tex
\usepackage{amsmath}
\usepackage{mathtools}
\setcounter{mchapter}{6}
\setcounter{mexercise}{0}

\begin{document}
  {
\beamertemplatenavigationsymbolsempty % suppress navigation on this (= first) slide
\begin{frame}[plain] % plain means: no header and footer on this (= first) slide
    \begin{textblock*}{0cm}(0.5cm, -0.7cm)
        \includegraphics[width=11.0cm]{common/logo_uni_tuebingen.png}
    \end{textblock*}
    \titlepage
    \begin{textblock*}{0cm}(0.1cm, -2.5cm)
        \textcolor{university_tuebingen}{\rule{11.8cm}{0.2cm}}
    \end{textblock*}
\end{frame}
}


  \section{Aufgaben zur Vertiefung}

  \subsection{Aufgabe a}
  	\begin{frame}
		\frametitle{Aufgabe a: Zahlenrechnen und Variablen}
          \begin{exercise}
              \sloppy          
           Berechnen Sie jeweils mit Matlab die folgenden Ausdrücke bzw. geben Sie diese Ausdrücke nach dem gezeigten Zeichen \texttt{$>>$} in Matlab ein.
Überlegen sie vorher, was Matlab mit dem Ausdruck macht!

		   \begin{footnotesize}     	
      	 	\begin{tabular}{ll}
        	  1. 4+7 & 16. sqrt(-3) \\
        	  2. v=4+7 & 17. b=3-4i \\ 
        	  3. v & 18. real(b) \\
        	  4. v+2\textasciicircum 3 & 19. imag(b) \\ 
        	  5. clear all & 20. abs(b) \\
        	  6. v & 21. round(24.3219)\\
        	  7. v+2\textasciicircum 3 & 22. round(24.9219)\\ 
          	  8. c=sin(5*$pi$/6)+1/3 & 23. round(24.9219)-24.9219 \\
          	  9. c & 24. round(round(24.9219)-24.9219)\\          	 
          	  10. format long & 25. round(100*24.9219)/100\\
          	  11. c & 26. help clear\\
          	  12. format short & 27. f=5 \\
          	  13. c & 28. f \\
          	  14. d & 29. clear f \\
          	  15. d=c\textasciicircum 2-5*c+sqrt(2)-1 & 30. f         	          	  
            \end{tabular} 
           \end{footnotesize}       	      
	            
		  \end{exercise}
   		\end{frame}

\subsection{Aufgabe b}
  	\begin{frame}
		\frametitle{Aufgabe b: 2D-Plots}
          \begin{exercise}
              \sloppy          
           Probieren Sie aus, was Matlab mit dem jeweiligen Ausdruck macht.
           
		  \begin{footnotesize}
		     \begin{enumerate}
      	 	 	\item \texttt{x=0:0.01:2*pi;}
				\item \texttt{plot(x,sin(x))}
				\item \texttt{plot(x,sin(x),’-’,x,cos(x),’-.’)}
				\item \texttt{axis([-0.2 2*pi+0.2 -1.2 1.2])}
				\item \texttt{legend(’Sin’,’Cos’)}
				\item \texttt{xlabel(’x’)}
				\item \texttt{ylabel(’f(x)’)}
				\item \texttt{x=[1 2.5 3 4 1];}
				\item \texttt{y=[1 -1 -2 1.5 0];}
				\item \texttt{plot(x,y,’P’)}
				\item \texttt{axis([-2 5 -3 3])}
      	 	 \end{enumerate}
		\end{footnotesize}         	 
      	 Erzeugen sie eigene Funktionen und Plots.
                
		  \end{exercise}
   		\end{frame}
 
\subsection{Aufgabe c}
  	\begin{frame}
		\frametitle{Aufgabe c: Subplot}
          \begin{exercise}
              \sloppy          
           \begin{itemize}
           	\item Erstellen Sie mit Hilfe von Unterabbildungen eine Übersicht, die räumliche Profile und Durchbruchskurven für punktartigen Stoffeintrag und stufenartige Anfangsverteilung darstellen (s. Skripte aus den Vorlesungen).
           	\item Erzeugen Sie hiervon eine schön aussehende Grafikdatei, die Sie z.B. für eine Powerpoint-Präsentation verwenden können.
           	\item Tipp: Bei einer Durchbruchskurve ist der Ort fest und die Zeit variabel, im Gegensatz zu räumlichen Profilen, bei denen der Ort variabel ist und die Zeit fest.
           \end{itemize}
                
		  \end{exercise}
   		\end{frame}


\end{document}
