\documentclass{beamer}

\usepackage[utf8]{inputenc}
\usepackage[T1]{fontenc}
\usepackage{textpos}
\usepackage{tikz}

\usetheme{Madrid}
\usecolortheme{beaver}

% Custom changes:
\setbeamertemplate{footline}[frame number]{}
\definecolor{university_tuebingen}{RGB}{165,30,55}
\setbeamercolor{frametitle}{fg=university_tuebingen, bg=white}
\setbeamercolor{title}{fg=university_tuebingen}

\addtobeamertemplate{frametitle}{}{
\begin{tikzpicture}[remember picture, overlay]
\node[anchor=north east,yshift=0cm] at (current page.north east)
{\includegraphics[width=3cm]{../common/logo_uni_tuebingen2.png}};
\end{tikzpicture}}



\author{Prof. Dr. Christiane Zarfl, Dipl.-Inf. Willi Kappler}
%\date{\today}
\date{}
%\institute{Universität Tübingen}
\institute{}


\matlabTitle{Hausaufgaben 2}

% Rechtschreibprüfung mit: aspell -l de -t --tex-check-comments -c lecture6/lecture6.tex
\usepackage{amsmath}
\usepackage{mathtools}
\setcounter{mchapter}{6}
\setcounter{mexercise}{0}

\begin{document}
  {
\beamertemplatenavigationsymbolsempty % suppress navigation on this (= first) slide
\begin{frame}[plain] % plain means: no header and footer on this (= first) slide
    \begin{textblock*}{0cm}(0.5cm, -0.7cm)
        \includegraphics[width=11.0cm]{common/logo_uni_tuebingen.png}
    \end{textblock*}
    \titlepage
    \begin{textblock*}{0cm}(0.1cm, -2.5cm)
        \textcolor{university_tuebingen}{\rule{11.8cm}{0.2cm}}
    \end{textblock*}
\end{frame}
}


  \section{Aufgaben zur Vertiefung (Teil 2)}

  \subsection{Grundwasserstand}
  	\begin{frame}
		\frametitle{Grundwasserstand für einen Brunnen\\ in Grundströmung}
		\begin{align*}
			h \left(x,y \right) &= C-I \cdot x + \frac{1}{2 \cdot \pi \cdot T} \cdot Q \cdot ln \left( \frac{r}{R} \right) \\
			r &= \sqrt{\left( x-x_{b} \right)^{2}+\left( y-y_{b} \right)^{2}}                
        \end{align*}  
		\begin{small}        
        \begin{itemize}
        	\item C: Integrationskonstante (= 0 m)
        	\item I: Hydraulischer Gradient ohne Brunnen (=1\%)
			\item Q: Förderrate des Brunnens (Q = 1000 m$^3$/Tag)
			\item T: Transmissivität des Grundwasserleiters (= 5*10$^{-3}$ m$^2$/s)
			\item r: Abstand zum Brunnen
			\item R: Reichweite des Brunnens (= 500 m)
			\item x$_b$, y$_b$:	Koordinaten des Brunnens (-50,0)
        \end{itemize}
        \end{small}
   	\end{frame}

 \begin{frame}
		\frametitle{Aufgabe a}
          \begin{exercise}
              \sloppy          
              \textbf{Grundwasserstand}
              \begin{itemize}
                \item Erstellen Sie einen Grundwassergleichenplan (Karte mit Höhenlinien des Grundwasserstandes \texttt{h(x,y)})
				\item Hinweise
				\begin{itemize}
					\item ln$\left(0\right)= -\infty$; deshalb ersetzen Sie alle Brunnenabstände < Brunnenradius (10 cm) durch den Brunnenradius. 
					\item \texttt{daspect[1 1 1]} verhindert Verzerrung im Plot.
				\end{itemize}
              \end{itemize}
          \end{exercise}
      \end{frame}

\begin{frame}
		\frametitle{Pseudocode Grundwasserstand}
		\begin{enumerate}
			\item Definiere alle Konstanten.
			\item Erzeuge regelmäßiges Gitter von (x,y)-Werten.
			\item Berechne Abstand aller Punkte zum Brunnen (Matrix \textbf{R}).
			\item Setze $R(R<r\_Brunnen)=r\_Brunnen$.
			\item Berechne Grundwasserpegel an allen Punkten (ergibt Matrix \textbf{H}).
			\item Erzeuge Höhenlinien-Grafik h(x,y).

		\end{enumerate}
			
   	\end{frame}


\subsection{Particle Tracking}
  	\begin{frame}
		\frametitle{Particle Tracking durch ein \\Geschwindigkeitsfeld}
		\begin{itemize}
			   \item Verfolge ein Teilchen auf dem Weg durch ein Geschwindigkeitsfeld $\frac{d\textbf{x}}{dt}=\textbf{v}\left(\textbf{x}\left(t \right) \right)$, \textbf{x} und \textbf{v} sind Vektoren          
               \item Geschwindigkeitsfeld: Brunnen plus Grundströmung
                 \begin{align*}
                     v_{x}(x,y) &= \frac{T}{n \cdot m}I+\frac{1}{2 \cdot \pi \cdot n \cdot m}\cdot Q \cdot \left( \frac{x-x_{b}}{r^{2}} \right) \\
         		     v_{y}(x,y) &= \frac{1}{2 \cdot \pi \cdot n \cdot m}Q \cdot \left( \frac{y-y_{b}}{r^{2}} \right)
         		 \end{align*}
          
                  \begin{itemize}
					\item m	Mächtigkeit des Grundwasserleiters (=10 m)
					\item n	Porosität (25 \%)
				  \end{itemize}
				 \item Integration durch explizites Euler-Verfahren: $\textbf{x} \left(t+\Delta t \right)=\textbf{x}(t)+\Delta t \cdot \textbf{v}(\textbf{x}(t))$

		\end{itemize}
   	\end{frame}

 \begin{frame}
		\frametitle{Aufgabe}
          \begin{exercise}
              \sloppy          
              \textbf{Particle Tracking}
              \begin{itemize}
                \item Setzen Sie Teilchen am Brunnenrand des Brunnens (Brunnenradius = 10 cm) ein.
				\item Verfolgen Sie das Teilchen, bis 1 Jahr vergangen ist.
				\item Grafische Ausgabe ist erwünscht.
				\item Hinweis:
				\begin{itemize}
					\item Eine feste Ortsschrittweite $\Delta x < r\_Brunnen$ statt fester Zeitschrittweite $\Delta t$ beschleunigt die Berechnung.

				\end{itemize}
              \end{itemize}
          \end{exercise}
      \end{frame}
      
      \begin{frame}
		\frametitle{Pseudocode Particle Tracking}
		%\begin{tiny}
          \lstinputlisting[language=pseudo]{pseudocode_particle_tracking.txt}

		%\end{tiny}
			
   	\end{frame}

\end{document}
