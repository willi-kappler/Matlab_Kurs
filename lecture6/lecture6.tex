\documentclass{beamer}

\usepackage[utf8]{inputenc}
\usepackage[T1]{fontenc}
\usepackage{textpos}
\usepackage{tikz}

\usetheme{Madrid}
\usecolortheme{beaver}

% Custom changes:
\setbeamertemplate{footline}[frame number]{}
\definecolor{university_tuebingen}{RGB}{165,30,55}
\setbeamercolor{frametitle}{fg=university_tuebingen, bg=white}
\setbeamercolor{title}{fg=university_tuebingen}

\addtobeamertemplate{frametitle}{}{
\begin{tikzpicture}[remember picture, overlay]
\node[anchor=north east,yshift=0cm] at (current page.north east)
{\includegraphics[width=3cm]{../common/logo_uni_tuebingen2.png}};
\end{tikzpicture}}



\author{Prof. Dr. Christiane Zarfl, Dipl.-Inf. Willi Kappler}
%\date{\today}
\date{}
%\institute{Universität Tübingen}
\institute{}


\matlabTitle{6. Differentialgleichungen}

% Rechtschreibprüfung mit: aspell -l de -t --tex-check-comments -c lecture6/lecture6.tex
\usepackage{amsmath}
\usepackage{mathtools}
\setcounter{mchapter}{6}
\setcounter{mexercise}{0}

\begin{document}
  {
\beamertemplatenavigationsymbolsempty % suppress navigation on this (= first) slide
\begin{frame}[plain] % plain means: no header and footer on this (= first) slide
    \begin{textblock*}{0cm}(0.5cm, -0.7cm)
        \includegraphics[width=11.0cm]{common/logo_uni_tuebingen.png}
    \end{textblock*}
    \titlepage
    \begin{textblock*}{0cm}(0.1cm, -2.5cm)
        \textcolor{university_tuebingen}{\rule{11.8cm}{0.2cm}}
    \end{textblock*}
\end{frame}
}


  \section{Einleitung}

  \subsection{Motivation}
  \begin{frame}
      \frametitle{Sie wissen bereits...}
      \begin{itemize}
          \item wie Sie durch Skripte Befehlsfolgen wiederverwertbar machen.
          \item wie Sie häufig vorkommende Berechnungen in Algorithmen formulieren.
          \item wie Sie eigene Funktionen zur Wiederverwertung von Berechnungsschritten erstellen.
      \end{itemize}

      \textit{Wie kann ich eigene Funktionen zur Lösung von Differentialgleichungen verwenden?}
  \end{frame}

  \begin{frame}
      \frametitle{Nach diesem sechsten Block...}
      \begin{itemize}
          \item können Sie mit Hilfe von selbst definierten Funktionen in Matlab bereits implementierte Lösungsmethoden für Differentialgleichungen verwenden.
       \end{itemize}
  \end{frame}


  \section{DGL}

  \subsection{DGL}
  \begin{frame}
      \frametitle{Differentialgleichungen (DGLn)}
      \begin{itemize}
        \item einfache DGL
           \begin{align*}
             \frac{dy}{dt} &= a \cdot y,\texttt{ mit y(0)= y$_{0}$ bei t=0}
           \end{align*}
        \item System von DGLn
            \begin{align*}
               \frac{dy_{1}}{dt} &= f_{1}\left( t,y_{1},...,y_{n} \right), y_{1}(t_{0})=y_{1,0} \\
               & \shortvdotswithin{=}
               \frac{dy_{n}}{dt} &= f_{1}\left( t,y_{1},...,y_{n} \right), y_{n}(t_{0})=y_{n,0} \\
               \Rightarrow \frac{d\textbf{y}}{dt} &= f\left( t,\textbf{y} \right), \textbf{y}(t_{0})=\textbf{y$_{0}$} \\
             \end{align*}
       
      \end{itemize}
  \end{frame}

\subsection{DGL mit Matlab}
  \begin{frame}
      \frametitle{Generische Lösung von DGLn mit Matlab}
      \begin{itemize}
		\item Matlab stellt Methoden zur DGL-Lösung bereit, z.B. \texttt{ode45, ode15s, $\cdots$}        
        \item Matlab braucht eine Funktion, in der die Veränderung \texttt{$\partial{x}/\partial{t}$} in Abhängigkeit der Zeit und des gegenwärtigen Wertes \texttt{x} errechnet wird
         \begin{itemize}
         	\item \texttt{function dxdt = velo(t,x)}\\
         	 $\cdots$ \\
         	 \texttt{dxdt = $\cdots$} \\
         	\item Input
         	  \begin{itemize}
         	    \item erstes Argument: Zeit t
         	    \item zweites Argument: Zustand x
         	  \end{itemize}
         	\item Output: Veränderung des Zustands mit der Zeit 
		 \end{itemize}         	   
        \item Matlab hat Funktionen zum Lösen von DGLn, die von einem Script (oder einer anderen Funktion) aufgerufen werden müssen \\
%        \begin{itemize*}
          \texttt{spanne = [tmin tmax]} \\
          \texttt{x0 = $\cdots$} \\
          \texttt{[t,x]=ode15s(@velo,spanne,x0)} \\
%       \end{itemize*}               
      \end{itemize}
  \end{frame}


  \secMexercise
    \begin{frame}
        \frameMexercise
        \begin{exercise}
            \sloppy
            \textbf{Radioaktiver Zerfall mit DGL-Löser}
            \begin{itemize}
              \item Schreiben Sie eine Funktion \texttt{radio.m} für die DGL zur Beschreibung des radioaktiven Zerfalls eines Isotops A mit 100 Bq als Startwert und einer Zerfallsrate von \texttt{0,005  [1/Jahr]}: \texttt{function dxdt=radio(t,x)}.
              \item Schreiben Sie ein aufrufendes Script. Plotten Sie den Zeitverlauf über einen anschaulich gewählten Zeitraum.
            \end{itemize}
        \end{exercise}
    \end{frame}

  \secMexercise
      \begin{frame}
          \frameMexercise
          \begin{exercise}
              \sloppy
              \textbf{Gekoppelte DGLn: Radioaktive Zerfallskette mit DGL-Löser}
              \begin{itemize}
                \item Sie möchten nun nicht nur das Isotop A selbst, sondern auch sein Zerfallsprodukt B betrachten, das wiederum zerfällt. Schreiben Sie eine Funktion \texttt{radiokette.m} für die DGLn zur Beschreibung dieser radioaktiven Zerfallskette. Was sind sinnvoll gewählte Startwerte für A und B? Wählen Sie wieder die Zerfallsrate von \texttt{0,005  [1/Jahr]} für A und \texttt{0,003  [1/Jahr]} für B. Was ändert sich für Ihre Funktion \texttt{function dxdt=radiokette(t,x)} im Vergleich zur vorhergehenden Aufgabe?
                \item Schreiben Sie wieder ein aufrufendes Script. Plotten Sie den Zeitverlauf von A und B.
              \end{itemize}
          \end{exercise}
      \end{frame}

  \secMexercise
      \begin{frame}
          \frameMexercise
          \begin{exercise}
              \sloppy
              \textbf{Zusatz: Particle Tracking durch ein Geschwindigkeitsfeld}
              \begin{itemize}
               \item Geschwindigkeitsfeld: Brunnen plus Grundströmung
                 \begin{displaymath}
                   \begin{split}
                     v_{x}(x,y) = \frac{T}{n \cdot m}I+\frac{1}{2 \cdot \pi \cdot n \cdot m}Q \cdot \left( \frac{x-x_{b}}{r^{2}} \right) \\
         		     v_{y}(x,y) = \frac{1}{2 \cdot \pi \cdot n \cdot m}Q \cdot \left( \frac{y-y_{b}}{r^{2}} \right)
         		    \end{split}
                   \end{displaymath}
          
                  \begin{itemize}
					\item m	Mächtigkeit des Grundwasserleiters (=10 m)
					\item n	Porosität (25 \%)
					\item I	Hydraulischer Gradient ohne Brunnen (=1\%)
					\item Q	Förderrate des Brunnens ( \texttt{Q = 1000 m$^3$/d})
					\item T	Transmissivität des Grundwasserleiters (\texttt{=5*10$^{-3}$ m$^2$/s})
					\item r	Abstand zum Brunnen
					\item xb,yb Koordinaten des Brunnens (-50,0)
				  \end{itemize}

                \end{itemize}
          \end{exercise}
      \end{frame}
      
      \begin{frame}
		\frametitle{Fortsetzung}
          \begin{exercise}
              \sloppy          
              \textbf{Zusatz: Particle Tracking durch ein Geschwindigkeitsfeld}
              \begin{itemize}
                \item Schreiben Sie eine Funktion velo.m für das Geschwindigkeitsfeld
\texttt{function v=velo(t,x)}
				\item Schreiben Sie ein aufrufendes Script mit Anfangspunkten wie in der Hausübung 
				\begin{itemize}
					\item Integrieren Sie über 14 Tage Laufzeit
					\item Plotten Sie die Trajektorien
				\end{itemize}
              \end{itemize}
          \end{exercise}
      \end{frame}
      
      

  \section{Finally}

  \subsection{Nützliches}
  \begin{frame}
      \frametitle{Nützliches}
      \begin{itemize}
          \item Sehr guter Matlab \urlLink{http://mo.mathematik.uni-stuttgart.de/kurse/kurs4/}{Online Kurs} mit Beispielen.
          \item Wie in den meisten anderen Dingen auch: Verwendung von Matlab wird ``vertrauter''/leichter mit der Übung.
      \end{itemize}
  \end{frame}



\end{document}
