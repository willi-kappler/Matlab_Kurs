\documentclass{beamer}

\usepackage[utf8]{inputenc}
\usepackage[T1]{fontenc}
\usepackage{textpos}
\usepackage{tikz}

\usetheme{Madrid}
\usecolortheme{beaver}

% Custom changes:
\setbeamertemplate{footline}[frame number]{}
\definecolor{university_tuebingen}{RGB}{165,30,55}
\setbeamercolor{frametitle}{fg=university_tuebingen, bg=white}
\setbeamercolor{title}{fg=university_tuebingen}

\addtobeamertemplate{frametitle}{}{
\begin{tikzpicture}[remember picture, overlay]
\node[anchor=north east,yshift=0cm] at (current page.north east)
{\includegraphics[width=3cm]{../common/logo_uni_tuebingen2.png}};
\end{tikzpicture}}



\author{Prof. Dr. Christiane Zarfl, Dipl.-Inf. Willi Kappler}
%\date{\today}
\date{}
%\institute{Universität Tübingen}
\institute{}


\matlabTitle{6. Differentialgleichungen}

% Rechtschreibprüfung mit: aspell -l de -t --tex-check-comments -c lecture6/lecture6.tex

\setcounter{mchapter}{6}
\setcounter{mexercise}{0}

\begin{document}
  {
\beamertemplatenavigationsymbolsempty % suppress navigation on this (= first) slide
\begin{frame}[plain] % plain means: no header and footer on this (= first) slide
    \begin{textblock*}{0cm}(0.5cm, -0.7cm)
        \includegraphics[width=11.0cm]{common/logo_uni_tuebingen.png}
    \end{textblock*}
    \titlepage
    \begin{textblock*}{0cm}(0.1cm, -2.5cm)
        \textcolor{university_tuebingen}{\rule{11.8cm}{0.2cm}}
    \end{textblock*}
\end{frame}
}


  \section{Einleitung}

  \subsection{Motivation}
  \begin{frame}
      \frametitle{Sie wissen bereits...}
      \begin{itemize}
          \item wie Sie durch Skripte Befehlsfolgen wiederverwertbar machen.
          \item wie Sie häufig vorkommende Berechnungen in Algorithmen formulieren.
          \item wie Sie eigene Funktionen zur Wiederverwertung von Berechnungsschritten erstellen.
      \end{itemize}

      \textit{Wie kann ich eigene Funktionen zur Lösung von Differentialgleichungen verwenden?}
  \end{frame}

  \begin{frame}
      \frametitle{Nach diesem sechsten Block...}
      \begin{itemize}
          \item können Sie mit Hilfe von selbst definierten Funktionen in Matlab bereits implementierte Lösungsmethoden für Differentialgleichungen verwenden.
          % Stimmt der zweite Punkt noch?
          \item können Sie mit Hilfe von selbst definierten (Fehler-)Funktionen in Matlab Parameterwerte bestimmen, die diese Funktion minimieren.
      \end{itemize}
  \end{frame}


  % Hier kannst du weitere Folien einfügen...
  \section{DGL}

  \subsection{DGL}
  \begin{frame}
      \frametitle{DGL}
      \begin{itemize}
        \item DGL mit Matlab...
      \end{itemize}
  \end{frame}


  \section{Finally}

  \subsection{Nützliches}
  \begin{frame}
      \frametitle{Nützliches}
      \begin{itemize}
          \item Sehr guter Matlab \urlLink{http://mo.mathematik.uni-stuttgart.de/kurse/kurs4/}{Online Kurs} mit Beispielen.
          \item Wie in den meisten anderen Dingen auch: Verwendung von Matlab wird ``vertrauter''/leichter mit der Übung.
      \end{itemize}
  \end{frame}



\end{document}
