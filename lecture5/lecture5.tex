\documentclass{beamer}

\usepackage[utf8]{inputenc}
\usepackage[T1]{fontenc}
\usepackage{textpos}
\usepackage{tikz}

\usetheme{Madrid}
\usecolortheme{beaver}

% Custom changes:
\setbeamertemplate{footline}[frame number]{}
\definecolor{university_tuebingen}{RGB}{165,30,55}
\setbeamercolor{frametitle}{fg=university_tuebingen, bg=white}
\setbeamercolor{title}{fg=university_tuebingen}

\addtobeamertemplate{frametitle}{}{
\begin{tikzpicture}[remember picture, overlay]
\node[anchor=north east,yshift=0cm] at (current page.north east)
{\includegraphics[width=3cm]{../common/logo_uni_tuebingen2.png}};
\end{tikzpicture}}



\author{Prof. Dr. Christiane Zarfl, Dipl.-Inf. Willi Kappler}
%\date{\today}
\date{}
%\institute{Universität Tübingen}
\institute{}


\matlabTitle{5. Funktionen}

% Rechtschreibprüfung mit: aspell -l de -t --tex-check-comments -c lecture5/lecture5.tex

\setcounter{mchapter}{5}
\setcounter{mexercise}{0}

\begin{document}
  {
\beamertemplatenavigationsymbolsempty % suppress navigation on this (= first) slide
\begin{frame}[plain] % plain means: no header and footer on this (= first) slide
    \begin{textblock*}{0cm}(0.5cm, -0.7cm)
        \includegraphics[width=11.0cm]{common/logo_uni_tuebingen.png}
    \end{textblock*}
    \titlepage
    \begin{textblock*}{0cm}(0.1cm, -2.5cm)
        \textcolor{university_tuebingen}{\rule{11.8cm}{0.2cm}}
    \end{textblock*}
\end{frame}
}


  \section{Einleitung}

  \subsection{Motivation}
  \begin{frame}
      \frametitle{Sie wissen bereits...}
      \begin{itemize}
          \item wie Sie durch Scripte Befehlsfolgen wiederverwertbar machen.
          \item wie Sie häufig vorkommende Berechnungen in Algorithmen formulieren.
      \end{itemize}

      \textit{Wie kann ich eigene Funktionen für verschiedene Eingaben erstellen und damit Berechnungsschritte wiederverwenden?}
  \end{frame}

  \begin{frame}
      \frametitle{Nach diesem fünften Block...}
      \begin{itemize}
          \item können Sie interaktive Eingaben anfordern und Ausgaben erstellen.
          \item können Sie eigene Funktionen definieren und z.B. in Scripten aufrufen/wiederverwenden.
      \end{itemize}
  \end{frame}

  \subsection{Ein- und Ausgabe}
  \begin{frame}
      \frametitle{Ein- und Ausgabe}
      \begin{itemize}
          \item Ausgabe eines Textes im Comand-Window:
            \begin{itemize}
                \item \matlabInput{\matlabLink{disp}} steht für ``display''
                \item Text-Strings sind von Hochkommata umschlossen
                \item \matlabInput{\matlabLink{disp}} versteht auch Vektoren: \matlabInput{disp(['He' 'llo' 'World'])}
            \end{itemize}
          \item Interaktive Eingabe:
          \begin{itemize}
              \item \matlabInput{a=\matlabLink{input}('Bitte Halbwertszeit [Tage]} \matlabInput{für Photoabbau eingeben: ')}
              \item erzeugt den Text auf dem Bildschirm und wartet, bis eine Eingabe abgeschlossen ist (Return beendet die Eingabe)
              \item Der eingegebene Wert steht dann in der Variable \matlabInput{a}
          \end{itemize}
      \end{itemize}
  \end{frame}

  \begin{frame}
      \frametitle{Ein- und Ausgabe Fortsetzung}
      \begin{itemize}
          \item Beispiel:
          \begin{itemize}
              \item \matlabInput{jn = input('Abbruch? (j/n) ','s')}
              \item Zusatzargument 's' erklärt, dass Ergebnis als Text-String zu lesen ist (selbst wenn der String aus Ziffern besteht)
              \item Umgang mit Fehlern der Nutzer $\Rightarrow$ typische Schleife, bis richtige Antwort kommt:
          \end{itemize}
          \lstinputlisting{example5_1.m}
      \end{itemize}
  \end{frame}

\end{document}
