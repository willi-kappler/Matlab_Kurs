\documentclass{beamer}

\usepackage[utf8]{inputenc}
\usepackage[T1]{fontenc}
\usepackage{textpos}
\usepackage{tikz}

\usetheme{Madrid}
\usecolortheme{beaver}

% Custom changes:
\setbeamertemplate{footline}[frame number]{}
\definecolor{university_tuebingen}{RGB}{165,30,55}
\setbeamercolor{frametitle}{fg=university_tuebingen, bg=white}
\setbeamercolor{title}{fg=university_tuebingen}

\addtobeamertemplate{frametitle}{}{
\begin{tikzpicture}[remember picture, overlay]
\node[anchor=north east,yshift=0cm] at (current page.north east)
{\includegraphics[width=3cm]{../common/logo_uni_tuebingen2.png}};
\end{tikzpicture}}



\author{Prof. Dr. Christiane Zarfl, Dipl.-Inf. Willi Kappler}
%\date{\today}
\date{}
%\institute{Universität Tübingen}
\institute{}


\matlabTitle{5. Funktionen}

% Rechtschreibprüfung mit: aspell -l de -t --tex-check-comments -c lecture5/lecture5.tex

\setcounter{mchapter}{5}
\setcounter{mexercise}{0}

\begin{document}
  {
\beamertemplatenavigationsymbolsempty % suppress navigation on this (= first) slide
\begin{frame}[plain] % plain means: no header and footer on this (= first) slide
    \begin{textblock*}{0cm}(0.5cm, -0.7cm)
        \includegraphics[width=11.0cm]{common/logo_uni_tuebingen.png}
    \end{textblock*}
    \titlepage
    \begin{textblock*}{0cm}(0.1cm, -2.5cm)
        \textcolor{university_tuebingen}{\rule{11.8cm}{0.2cm}}
    \end{textblock*}
\end{frame}
}


  \section{Einleitung}

  \subsection{Motivation}
  \begin{frame}
      \frametitle{Sie wissen bereits...}
      \begin{itemize}
          \item wie Sie durch Skripte Befehlsfolgen wiederverwendbar machen.
          \item wie Sie häufig vorkommende Berechnungen in Algorithmen formulieren.
      \end{itemize}

      \textit{Wie kann ich eigene Funktionen für verschiedene Eingaben erstellen und damit Berechnungsschritte wiederverwenden?}
  \end{frame}

  \begin{frame}
      \frametitle{Nach diesem fünften Block...}
      \begin{itemize}
          \item können Sie interaktive Eingaben anfordern und Ausgaben erstellen.
          \item können Sie eigene Funktionen definieren und z.B. in Skripten aufrufen/wiederverwenden.
      \end{itemize}
  \end{frame}

  \subsection{Ein- und Ausgabe}
  \begin{frame}
      \frametitle{Ein- und Ausgabe}
      \begin{itemize}
          \item Ausgabe eines Textes im Command-Window:
            \begin{itemize}
                \item \matlabInput{\matlabLink{disp}} steht für ``display''
                \item Text-Strings sind von Hochkommata umschlossen
                \item \matlabInput{\matlabLink{disp}} versteht auch Vektoren: \matlabInput{disp(['He' 'llo' 'World'])}
            \end{itemize}
          \item Interaktive Eingabe:
          \begin{itemize}
              \item \matlabInput{a = \matlabLink{input}('Bitte Halbwertszeit [Tage]} \matlabInput{für Photoabbau eingeben: ')}
              \item erzeugt den Text auf dem Bildschirm und wartet, bis eine Eingabe abgeschlossen ist (Return beendet die Eingabe).
              \item Der eingegebene Wert steht dann in der Variable \matlabInput{a}.
          \end{itemize}
      \end{itemize}
  \end{frame}

  \begin{frame}
      \frametitle{Ein- und Ausgabe (Fortsetzung)}
      \begin{itemize}
          \item Beispiel:
          \begin{itemize}
              \item \matlabInput{jn = input('Abbruch? (j/n) ','s')}
              \item Zusatzargument 's' erklärt, dass Ergebnis als Text-String zu lesen ist (selbst wenn der String aus Ziffern besteht).
              \item Umgang mit Fehlern der Nutzer $\Rightarrow$ typische Schleife, bis richtige Antwort kommt:
          \end{itemize}
          \lstinputlisting{example5_1.m}
      \end{itemize}
  \end{frame}


  \secMexercise
  \begin{frame}
      \frameMexercise
      \begin{exercise}
          \sloppy
          Überlagerung von zwei Sinusschwingungen (schwing.m)
          \begin{itemize}
            \item Darzustellende Funktion:
            \item $ y(x) = a_{1}\sin(fx + \phi_{1}) + a_{2}\sin(fx + \phi_{2})$
            \item Interaktive Eingabe:
            \begin{itemize}
              \item gemeinsame Frequenz \texttt{f}
              \item Amplitude der ersten Komponente \texttt{a1}
              \item Amplitude der ersten Komponente \texttt{a2}
              \item Phasenwinkel der ersten Komponente \texttt{phi1}
              \item Phasenwinkel der zweiten Komponente \texttt{phi2}
            \end{itemize}
            \item Graphische Ausgabe im Intervall $[-\pi,+\pi]$
          \end{itemize}
      \end{exercise}
  \end{frame}

  \section{Funktionen}

  \subsection{Skripte und Funktionen}
  \begin{frame}
      \frametitle{Skripte und Funktionen}
      \begin{itemize}
        \item ``Idee'' eines Skriptes: ersetzt manuelle Eingabe einer Befehlsfolge:
        \begin{itemize}
          \item Variablen stehen auch nach Ausführung des Skriptes zu Verfügung.
          \item Wenn Variablenwerte im Skript verändert werden, sind sie auch außerhalb des Skriptes verändert.
          \item Ein Skript kann ein anderes Skript aufrufen.
          \begin{itemize}
            \item mit gemeinsamer Nutzung des Arbeitsspeichers.
          \end{itemize}
          \item Nachteile eines Skriptes:
          \begin{itemize}
            \item Allgemeiner Arbeitsspeicher wird mit Zwischengrößen ``zugemüllt''.
            \item Variablenbezeichnung muss über alle Skripte hinweg konsistent sein (\alert{Vorsicht!} Seiteneffekte).
          \end{itemize}
        \end{itemize}
      \end{itemize}
  \end{frame}

  \begin{frame}
      \frametitle{Skripte und Funktionen (Fortsetzung)}
      \begin{itemize}
        \item ``Idee'' einer \textbf{Funktion}: ``Input $\Rightarrow$ Output''-Beziehung:
        \begin{itemize}
          \item Von der Außenwelt ist nur das bekannt, was als Eingabeargument(e) übergeben wird.
          \item Der Außenwelt wird lediglich das Ergebnis (Ausgabeargument(e)) mitgeteilt.
          \item Innenwelt der Funktion ist von außen nicht einsehbar.
          \begin{itemize}
            \item Eine Funktion hat ihren eigenen Speicherbereich.
          \end{itemize}
        \end{itemize}
      \end{itemize}
  \end{frame}

  \subsection{Vorteile}
  \begin{frame}
      \frametitle{Vorteile von Funktionen}
      \begin{itemize}
        \item Modularisierung:
        \begin{itemize}
          \item Unterteile ein großes Projekt in viele Einzelschritte und stelle Methoden für die Einzelschritte zur Verfügung.
          \item Eigener Namensraum, in verschiedenen Funktionen kann es die lokale Variable \texttt{i} oder \texttt{t} geben.
        \end{itemize}
        \item Wiederverwendbarkeit:
        \begin{itemize}
          \item Identischer Einzelschritt kann mehrfach, auch in anderen Projekten, verwendet werden.
        \end{itemize}
        \item \urlLink{https://en.wikipedia.org/wiki/Encapsulation_\%28computer_programming\%29}{Einkapselung}:
        \begin{itemize}
          \item Wenn eine Funktion funktioniert, interessiert mich nicht ihr Innenleben, und ich will nicht mit Zwischenergebnissen belästigt werden.
          \item \urlLink{https://en.wikipedia.org/wiki/Black_box}{Black Box} Prinzip (s.a. \urlLink{https://en.wikipedia.org/wiki/Information_hiding}{Information Hiding})
        \end{itemize}
      \end{itemize}
  \end{frame}

  \subsection{Schnittstelle}
  \begin{frame}
      \frametitle{Funktionen erfordern eine Schnittstelle}
      Funktionen - Definition in Matlab:
      \begin{itemize}
        \item Definiert in der ersten Zeile
        \item \matlabInput{\matlabLink{function} [out1,out2] = myfunction(in1,in2)}
        \begin{itemize}
          \item Erstes Wort = Schlüsselwort \matlabInput{\matlabLink{function}}
          \item Liste der \textbf{Eingabeparameter} (in eckigen Klammern, also als Vektor).
          \item Name der Funktion mit Liste der \textbf{Ausgabeparameter} in runden Klammern.
        \end{itemize}
        \item Die Schnittstelle ist definiert über die Reihenfolge der Argumente, nicht über die Namen.
        \begin{itemize}
          \item \textbf{Vorteil}: Variablennamen im Innern können sich von den Variablennamen außen unterscheiden.
          \item \textbf{Nachteil}: Man muss sich die Reihenfolge merken.
          \item Voraussetzung für allgemeine Wiederverwendbarkeit.
        \end{itemize}
      \end{itemize}
  \end{frame}

  \subsection{Hilfetexte}
  \begin{frame}
      \frametitle{Erstellung eigener Hilfetexte}
      \begin{itemize}
        \item Kommentarzeilen direkt unter der Kopfzeile werden bei \matlabInput{\matlabLink{help} Funktionsname} ausgegeben
        (z.B. \matlabInput{\matlabLink{help} myfunction}):
        \lstinputlisting{example5_2.m}
      \end{itemize}
  \end{frame}

  \secMexercise
  \begin{frame}
      \frameMexercise
      \begin{exercise}
          \sloppy
          Bogenmaß in Grad umwandeln:
          \begin{itemize}
            \item Funktion \texttt{phi\_deg = r2d(phi)} soll einen Winkel, der in Bogenmaß ($0-2\pi$ ) gegeben ist, in Gradmaß (0-360$^{\circ}$) umwandeln.
          \end{itemize}
      \end{exercise}
  \end{frame}

  \secMexercise
  \begin{frame}
      \frameMexercise
      \begin{exercise}
          \sloppy
          Mittelwert und Standardabweichung:
          \begin{itemize}
            \item Funktion \texttt{[m,s] = mittel\_standardabw(x)} soll Mittelwert und Standardabweichung des Vektors \texttt{x} berechnen. Verwenden Sie für Ihre Funktion einfach die in Matlab implementierten Befehle \texttt{mean} und \texttt{std}.
            \item Schreiben Sie zusätzlich ein Skript, das \textbf{interaktiv} nach der Anzahl \texttt{n} normal verteilter Zahlen mit vorgegebenem
            Mittelwert \texttt{m\_init} und Standardabweichung \texttt{s\_init} fragt, diese generiert und die Funktion
            \texttt{mittel\_standardabw} aufruft und ausgibt.
            \item Tipp: normal verteilte Zufallszahlen generieren: \texttt{x = m\_init + s\_init * randn(n,1);}
          \end{itemize}
      \end{exercise}
  \end{frame}

  \secMexercise
  \begin{frame}
      \frameMexercise
      \begin{exercise}
          \sloppy
          \begin{itemize}
            \item Abstand zwischen einem Punkt \texttt{($x_{0}$, $y_{0}$)} zu einem (oder vielen) Punkt(en) \texttt{(X,Y)}:
            \begin{displaymath}
              r = \sqrt{(x - x_{0})^{2} + (y - y_{0})^{2}}
            \end{displaymath}
            \item Funktionskopf: \texttt{r = dist(x0,y0,X,Y)}
            \item Schreiben Sie die Funktion \texttt{dist} in eine Datei dist.m.
            \item Schreiben Sie ein Skript, in welchem Sie:
            \begin{itemize}
              \item den Punkt \texttt{(x0, y0)} interaktiv eingeben.
              \item die Matrizen \textbf{X} und \textbf{Y} auf einem regelmäßigen Gitter erzeugen (\texttt{meshgrid}).
              \item Ihre Abstandsfunktion \texttt{dist} aufrufen.
              \item Das Ergebnis als \texttt{contour}-Grafik darstellen.
            \end{itemize}
          \end{itemize}
      \end{exercise}
  \end{frame}

  \section{Ausblick}

  \subsection{Ausblick}
  \begin{frame}
      \frametitle{Ausblick}
      \begin{itemize}
        \item Wie kann ich eigene Funktionen nutzen, um Differentialgleichungen mit Matlab zu lösen?
      \end{itemize}
  \end{frame}



\end{document}
