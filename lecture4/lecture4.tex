\documentclass{beamer}

\usepackage[utf8]{inputenc}
\usepackage[T1]{fontenc}
\usepackage{textpos}
\usepackage{tikz}

\usetheme{Madrid}
\usecolortheme{beaver}

% Custom changes:
\setbeamertemplate{footline}[frame number]{}
\definecolor{university_tuebingen}{RGB}{165,30,55}
\setbeamercolor{frametitle}{fg=university_tuebingen, bg=white}
\setbeamercolor{title}{fg=university_tuebingen}

\addtobeamertemplate{frametitle}{}{
\begin{tikzpicture}[remember picture, overlay]
\node[anchor=north east,yshift=0cm] at (current page.north east)
{\includegraphics[width=3cm]{../common/logo_uni_tuebingen2.png}};
\end{tikzpicture}}



\author{Prof. Dr. Christiane Zarfl, Dipl.-Inf. Willi Kappler}
%\date{\today}
\date{}
%\institute{Universität Tübingen}
\institute{}


\matlabTitle{4. Schleifen}

% Rechtschreibprüfung mit: aspell -l de -t --tex-check-comments -c lecture4/lecture4.tex

\setcounter{mchapter}{4}
\setcounter{mexercise}{0}

\begin{document}
  {
\beamertemplatenavigationsymbolsempty % suppress navigation on this (= first) slide
\begin{frame}[plain] % plain means: no header and footer on this (= first) slide
    \begin{textblock*}{0cm}(0.5cm, -0.7cm)
        \includegraphics[width=11.0cm]{common/logo_uni_tuebingen.png}
    \end{textblock*}
    \titlepage
    \begin{textblock*}{0cm}(0.1cm, -2.5cm)
        \textcolor{university_tuebingen}{\rule{11.8cm}{0.2cm}}
    \end{textblock*}
\end{frame}
}


  \section{Einleitung}

  \subsection{Motivation}
  \begin{frame}
      \frametitle{Sie wissen bereits...}
      \begin{itemize}
          \item wie Sie durch Skripte Befehlsfolgen wiederverwendbar machen.
      \end{itemize}

      \textit{Wie kann ich häufig vorkommende Berechnungen/Abläufe automatisieren?}
  \end{frame}

  \begin{frame}
      \frametitle{Nach diesem vierten Block...}
      \begin{itemize}
          \item können Sie Berechnungen ``umgangssprachlich'' als Algorithmus formulieren.
          \item können Sie in Matlab Algorithmen implementieren und verwenden dabei sicher die Hilfsmittel der
          \begin{itemize}
              \item logischen Operatoren
              \item Wenn-Dann-Anweisungen
              \item While- und For-Schleifen
          \end{itemize}
      \end{itemize}
  \end{frame}

  \section{Programmieren}

  \subsection{Vorgehen}
  \begin{frame}
      \frametitle{Matlab als Programmiersprache}
      \begin{itemize}
          \item \textbf{Ziel}: Lösen einer (beliebig) komplizierten Berechnungsaufgabe
          \item \textbf{Vorgehen}:
          \begin{enumerate}
              \item Genaue Beschreibung der Aufgabe
              \item Formulierung eines \textbf{Algorithmus} (Schritt-für-Schritt Berechnungsanweisung) in Form eines Ablaufplans/``Kochrezepts'' oder Pseudocode
              \item Umsetzung in Programmiersprache
              \item Testen des Programms, evtl. zurück zu Punkt 2, oder 3
              \item Wartung und Pflege des Programms während der Nutzung
          \end{enumerate}
          \item Setzt voraus, dass Möglichkeiten der Programmiersprache bekannt sind
          \item In diesem Kurs geht es hauptsächlich um Punkt 3
      \end{itemize}
  \end{frame}


  \subsection{Pseudocode}
  \begin{frame}
      \frametitle{Pseudocode}
      \begin{itemize}
          \item Sprachliche Mischung aus natürlicher Sprache, mathematischer Notation und einer höheren Programmiersprache
          \item Dient genauer Beschreibung des Algorithmus
          \item Ist für Menschen leicht verständlich
          \begin{itemize}
              \item kann aber vom Computer (noch) nicht ausgeführt werden
              \item ist in den meisten Fällen keine Programmiersprache
              \item orientiert sich aber oft an ``echten'' Programmiersprachen
          \end{itemize}
          \item Soll Algorithmen verständlich und klar ausdrücken, ohne auf die Eigenheiten einer Programmiersprache Rücksicht nehmen zu müssen
          \item Umgangssprache hilft, Verfahrensschritte zu verdeutlichen
          \begin{itemize}
            \item \texttt{``durchlaufe das Feld a mit Index i''}
            \item \texttt{``vertausche die Inhalte der Variablen x und y''}
          \end{itemize}
      \end{itemize}
  \end{frame}

  \subsection{Beispiel: Telefonieren}
  \begin{frame}
      \frametitle{Beispiel: Telefonieren}
      \begin{itemize}
          \item Telefoniervorgang als Pseudocode:
          \lstinputlisting[language=pseudo]{pseudocode.txt}
      \end{itemize}
    \end{frame}

    \subsection{Flussdiagramm}
    \begin{frame}
        \frametitle{Flussdiagramm}
        \begin{itemize}
          \item Telefoniervorgang als Flussdiagramm:
        \end{itemize}
        \begin{tikzpicture}[node distance = 2.5cm]
          \node [block] (init) {Hörer abnehmen};
          \node [decision, right of=init] (free) {Frei-zeichen};
          \node [block, right of=free] (dial) {wähle};
          \node [decision, right of=dial] (long) {langes Tuten};
          \node [block, right of=long] (wait) {auf Antwort warten};
          \node [decision, below of=long] (short) {kurzes Tuten};
          \node [block, right of=short] (hangup) {lege auf};
          \node [block, below of=free, text width=5em] (check) {lege auf und überprüfe Anschluss};
          \path [line] (init) -- (free);
          \path [line] (free) -- node [above] {ja} (dial);
          \path [line] (free) -- node [left] {nein} (check);
          \path [line] (dial) -- (long);
          \path [line] (long) -- node [above] {ja} (wait);
          \path [line] (long) -- node [left] {nein} (short);
          \path [line] (short) -- node [above] {ja} (hangup);
        \end{tikzpicture}
      \end{frame}

      \secMexercise
      \begin{frame}
          \frameMexercise
          \begin{exercise}
              \sloppy
              Schreiben Sie einen Pseudocode für die Berechnung des Mittelwerts \textit{m} eines Zahlenvektors \textbf{x} \\
              \begin{displaymath}
                m = \frac{1}{n} \sum_{i=1}^{n}X_{i}
              \end{displaymath}
          \end{exercise}
      \end{frame}

      \subsection{Logische Operationen}
      \begin{frame}
          \frametitle{Logische Operationen}
          \begin{itemize}
            \item Vergleich zwischen zwei \textbf{Zahlen}:
            \begin{itemize}
              \item \matlabInput{a=(3>1)} $\Rightarrow$ erzeugt logische Variable \texttt{a} mit Wert \matlabOutput{true (1)}
              \item \matlabInput{2>=4} $\Rightarrow$ \matlabOutput{false (0)}
              \item \matlabInput{1<pi} $\Rightarrow$ \matlabOutput{true (1)}
              \item \matlabInput{3==pi} $\Rightarrow$ \matlabOutput{false (0)}
              \item vergleichendes ``ist gleich'' durch \texttt{``==''}, Zuweisung durch \texttt{``=''}
            \end{itemize}
            \item Vergleich zwischen zwei \textbf{Vektoren}:
            \begin{itemize}
              \item \matlabInput{[1:4]<[5:-2:-1]} $\Rightarrow$ \matlabOutput{[true,true,false,false]}
            \end{itemize}
            \item Logisches ``und'' durch ``\string&'':
            \begin{itemize}
              \item \matlabInput{(x>1)\string&(x<5)} $\Rightarrow$ wahr für alle Werte von \texttt{x} zwischen 1 und 5
            \end{itemize}
            \item Logisches ``oder'' durch ``\string|'':
            \begin{itemize}
              \item \matlabInput{(x>1)|(x<5)} $\Rightarrow$ ist immer wahr
            \end{itemize}
            \item Logische Negation durch ``\string~'':
            \begin{itemize}
              \item \matlabInput{a~=5} $\Rightarrow$ wahr, wenn a $\neq$ 5
            \end{itemize}
          \end{itemize}
      \end{frame}

      \begin{frame}
          \frametitle{Logische Operationen als Auswahlindizes}
          \begin{itemize}
            \item Die Benutzung logischer Operatoren als Auswahlindizes (Filter) ist eine angenehme Besonderheit von Matlab.
            \item Erzeuge einen x-Vektor zwischen 0 und 10. Wähle alle x-Werte für $x^2>=4$ und $x^2<=64$ aus.
            \item In Matlab sehr einfach zu realisieren:
            \begin{itemize}
              \item \matlabInput{x = [0:1:10]}
              \item \matlabInput{x(x.\string^2>=4 \string& x.\string^2<=64)}
              \item Gibt \matlabOutput{[2 3 4 5 6 7 8]} zurück
            \end{itemize}
          \end{itemize}
        \end{frame}

      \subsection{Verzweigung}
      \begin{frame}
          \frametitle{IF/THEN/ELSE Anweisungen}
          \begin{itemize}
            \item Beispiel: Ratengesetz nur dann auswerten, wenn die Konzentration größer null ist:
            \lstinputlisting{example4_1.m}
            \item Allgemein formuliert erlaubt Matlab (siehe \matlabInput{doc \matlabLink{if}}):
            \lstinputlisting{example4_2.m}
          \end{itemize}
      \end{frame}

      \subsection{Schleifen}
	  \begin{frame}
          \frametitle{Schleifen: Pseudocode WHILE}
           \begin{itemize}
             \item Studieren als Pseudocode:
             \lstinputlisting[language=pseudo]{pseudocode_schleife_while.txt}
          \end{itemize}
      \end{frame}      
      
      
      \begin{frame}
          \frametitle{WHILE-Schleifen}
          \begin{itemize}
            \item Wird wiederholt, solange Bedingung erfüllt ist:
            \lstinputlisting{example4_3.m}
          \end{itemize}
      \end{frame}

	  \begin{frame}
          \frametitle{Schleifen: Pseudocode FOR}
           \begin{itemize}
             \item Vorlesungsbesuch als Pseudocode:
             \lstinputlisting[language=pseudo]{pseudocode_schleife_for.txt}
          \end{itemize}
      \end{frame}
      
      
      \begin{frame}
          \frametitle{FOR-Schleifen}
          \begin{itemize}
            \item Schleife mit vorgegebener Anzahl an Wiederholungen:
            \lstinputlisting{example4_4.m}
            \item Bei jedem Schleifendurchgang wird dem Index \texttt{i} der nächste Wert des Vektors \texttt{[1:10]} zugewiesen
            \item Muss nicht  ein einfacher Zähler sein:
            \lstinputlisting{example4_5.m}
          \end{itemize}
      \end{frame}

      \begin{frame}
          \frametitle{While oder For ?}
          \begin{columns}[t]
            \begin{column}{5cm}
              While Schleife
              \begin{itemize}
                \item Gut, wenn vorher nicht genau bekannt ist, wann der Abbruch erfolgen muss.
                \item Erlaubt kompliziertes Fortführungskriterium.
                \item \alert{Gefahren}:
                \begin{itemize}
                  \item verpasster Einstieg (Typ \matlabInput{while 0==1})
                  \item Endlosschleife, weil Kriterium immer erfüllt (Typ \matlabInput{while 1==1})
                \end{itemize}
              \end{itemize}
            \end{column}

            \begin{column}{6cm}
              For Schleife
              \begin{itemize}
                \item Gut, wenn Anzahl der Durchgänge vorher festliegt.
                \item Ein/Ausstieg gewährleistet.
                \item Zusätzliche Verlängerung nicht möglich.
                \item Vorzeitiger Ausstieg mit \matlabInput{\matlabLink{break}}-Befehl möglich:
                \lstinputlisting{example4_6.m}
              \end{itemize}
            \end{column}
          \end{columns}
      \end{frame}

      \secMexercise
      \begin{frame}
          \frameMexercise
          \begin{exercise}
              \sloppy
              \begin{itemize}
                \item Schreiben Sie ein Matlab-Programm zur Berechnung des Mittelwerts eines Zufallsvektors unter Verwendung von Schleifen
                \item Schreiben Sie denselben Code vektorisiert (\texttt{sum(x)} berechnet die Summe des Vektors \textbf{x}).
                \item Vergleichen Sie Ihr Ergebnis mit dem Matlab-Befehl \texttt{mean}!
                \item \textbf{Zusatz}: Die Befehle \texttt{tic; ... toc}; messen die Zeit. Vergleichen Sie die Laufzeit ihres Programms und die des
                Matlab-Befehls \texttt{mean} (für große Vektoren)!
              \end{itemize}
          \end{exercise}
      \end{frame}

      \subsection{Ausblick}
      \begin{frame}
          \frametitle{Ausblick}
          \begin{itemize}
            \item Hausaufgaben zur Vertiefung
            \begin{itemize}
              \item Sie können umgangssprachlich formulierte Anweisungen in ein Matlab-Programm ``übersetzen''.
              \item Sie implementieren eine komplexe Berechnung in einem Skript und stellen diese grafisch in einem Oberflächenplot dar.
            \end{itemize}
            \item Nächste Einheit - noch mehr ``Recycling'':
            \begin{itemize}
              \item Implementieren von eigenen Funktionen in Matlab
            \end{itemize}
          \end{itemize}
      \end{frame}


\end{document}
